\documentclass[fontsize=12pt, paper=a4]{scrartcl}

\usepackage{fontspec}
\setmainfont{TeX Gyre Termes}
%\setsansfont{Arial}

\usepackage[ngerman]{babel}
\usepackage[hidelinks]{hyperref}
\usepackage{csquotes}
\usepackage{listings}

\usepackage[backend = biber, %
style = apa, %
bibencoding = utf8, %
sorting = nyt, %
sortlocale = auto, %
maxcitenames = 2, %
maxbibnames = 99, %
language = auto, %
dashed = false, %
hyperref = true]{biblatex}

\addbibresource{gesis_panel_bibliography.bib} %% you can add multiple 
%%bibliographies


\title{Das GESIS Panel -- die Bibliographie}
\author{All die verehrten Nutzer*innen des GESIS Panels}
\date{\today}

\begin{document}

\maketitle

\section{Die Wahrheit}

Das GESIS Panel ist ganz toll \parencite{bosnjak_establishing_2017}. Das sagt 
jede \parencite{struminskaya_respondent_2016}. Besonders hervorheben ist die 
Arbeit von \textcite[616]{wahlig_panelmanagement:_2018}. Aber auch die Arbeiten 
von \textcites{ackermann_stealth_2018, 
bosnjak_establishing_2017,blom_comparison_2016,danner_development_2016} sind 
ganz ausgezeichnet.

Wiederholen wir doch noch einmal den letzten Satz, jetzt aber mit gefakten 
Seitenzahlen: Aber auch die Arbeiten von 
\textcites[123]{ackermann_stealth_2018}[34]{bosnjak_establishing_2017}
{blom_comparison_2016}[vgl. besonders][897]{danner_development_2016} sind ganz 
ausgezeichnet


\section{Die Quelle der Wahrheit}

\begin{footnotesize}
\lstinputlisting[language={[LaTeX]TeX}]{biblatex-sample.tex}	
\end{footnotesize}


\printbibliography

\end{document}
